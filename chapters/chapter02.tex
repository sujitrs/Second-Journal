%\section{Statement of the Problem}

Significant efforts have been made to standardize self-sovereign identity (SSI) and related technologies. These efforts aim to create a common framework and set of protocols that enable interoperability and widespread adoption of SSI systems. 

\par Here are some of the key standardization initiatives and organizations involved in SSI:

\subsubsection{Decentralized Identity Foundation (DIF)} DIF is a consortium of organizations working on the development of open standards for decentralized identity technologies, including SSI. They focus on creating interoperable specifications, protocols, and reference implementations to enable secure and privacy-preserving identity solutions.

\subsubsection{World Wide Web Consortium (W3C)} The W3C has been actively working on standards for decentralized identity, particularly through the Verifiable Credentials and Decentralized Identifiers (DIDs) Working Groups. These groups are developing specifications that define how digital credentials can be issued, verified, and exchanged in a decentralized and interoperable manner.

\subsubsection{Hyperledger Indy and Aries} These are open-source projects under the Hyperledger umbrella that focus on SSI and interoperable identity solutions. Hyperledger Indy provides the underlying distributed ledger technology, while Hyperledger Aries offers a toolkit for building interoperable identity agents.

\subsubsection{Trust over IP Foundation (ToIP)} ToIP is a global project that aims to define a new standard for trustworthy and interoperable digital identities. It brings together various stakeholders, including governments, corporations, and non-profit organizations, to collaborate on the development of open standards and protocols for decentralized identity.

\subsubsection{OpenID Foundation} While traditionally focused on identity solutions, the OpenID Foundation has also been exploring the development of standards for decentralized identity and SSI. They are working on extensions to the OpenID Connect protocol to support verifiable credentials and decentralized identifiers.

\subsubsection{ISO/IEC Standards} The International Organization for Standardization (ISO) and the International Electrotechnical Commission (IEC) have also been involved in standardization efforts related to digital identity. ISO/IEC JTC 1/SC 27 has been working on various standards related to IT security techniques, including those that pertain to identity management.

\subsubsection{Government Initiatives} Some governments and regulatory bodies have shown interest in standardizing SSI for various use cases, such as digital driver's licenses or government-issued credentials. These initiatives may involve collaboration with international standards organizations and industry stakeholders.

