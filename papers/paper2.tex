\section{Paper 2}
    %Begin Table formatting
    \begin{center}
    \begin{tabular}{ | m{5em} | m{25em} |} 
      \hline
      Author(s) Name & Seungjoo Lim;Min-Hyung Rhie;DongYeop Hwang;Ki-Hyung Kim \\ 
      \hline
      Paper Name & A Subject-Centric Credential Management Method based on the Verifiable Credentials \\ 
      \hline
      Publication Year & 2021 \\ 
      \hline
    \end{tabular}
    \end{center}
    %End Table formatting
    In \cite{9333857}, The paper tries to focus on subject centric control over holder centric control. The provision is made in W3C Verifiable Credentials(VC) Specifications for transfer of VC. The author argues that in case if a human subject transfers VC to a holder then subject loses complete control over the shared/transfered VC. To maintain the control over transfered VC, subject can encrypt the VC using its private key. So that as and when holder shares the VC representation to Verifier then verifier has to contact subject to decrypt the VC information; this way human subjects can be in total control of their VCs even if it is transferred. Subject needs to be in control of VC makes sense only if subject are humans \cite{9333857}.
