\section{Paper 9}
    %Begin Table formatting
    \begin{center}
    \begin{tabular}{ | m{5em} | m{25em} |} 
      \hline
      Author(s) Name &  Rajat Singh Verma;B. R. Chandavarkar;Pradeep Nazareth\\ 
      \hline
      Paper Name &  Mitigation of hard-coded credentials related attacks using QR code and secured web service for IoT\\ 
      \hline
      Publication Year &  2019 \\ 
      \hline
    \end{tabular}
    \end{center}
    %End Table formatting
    In \cite{8944592}, Internet of Things (IoT) devices with their limited available resources (power, compute, storage) becomes vulnerable to security attacks because of Hard-coded credentials such as clear text log-in id and password provided by the IoT manufacturers and unsecured ways of remotely accessing IoT devices are the major security concerns of industry and academia. In such scenarios, a lightweight security algorithm up to some extent can minimize the risk. This paper proposes one such approach using \ac{QR} code to mitigate hard-coded credentials related attacks such as Mirai malware, wreak havoc, etc. The QR code based approach provides non-clear text unpredictable login id and password. Further, this paper also proposes a secured way of remotely accessing IoT devices through modified https. The proposed algorithms are implemented and verified using Raspberry Pi 3 model B. 
    \par But the use of central server for authentication and authorisation make it single point of failure and cannot be deployed to use as is; there is need to address this issue by either having high availability for central server for authentication or having decentralised approach for authentication (AuthN) and authorisation (AuthZ).